\documentclass[a5paper,12pt]{article}
\usepackage[top=1.5cm,bottom=3cm,inner=1.8cm,outer=2cm]{geometry}
\usepackage[osf]{coelacanth}
\usepackage[ngerman]{babel}
\usepackage[utf8]{inputenc}
\usepackage{tikz}
\begin{document}

\begin{minipage}{.65\textwidth}

\vspace{.5cm}


\tikz\draw (0,0) -- (4,0);\linebreak
Name des Spielers

\begin{tabular}{ll}
\tikz\draw (0,0) -- (3,0);\linebreak
Name des Charakters
&
\tikz\draw (0,0) -- (3,0);\linebreak
Gesinnung \\

\tikz\draw (0,0) -- (3,0);\linebreak
Klasse
&
\tikz\draw (0,0) -- (3,0);\linebreak
Klasse \\
\end{tabular}

\begin{tabular}{cc}
% Trefferpunkte
\tikz [baseline = 0.6ex]
\draw [scale=0.7] (-1,0.3) .. controls (-1,1) and (-0.2,1) .. (0,0.5)
      (0,0.5) .. controls (0.2,1) and (1,1) .. (1,0.3)
      (1,0.3) .. controls (1,-0.4) and (0.5,-0.8) .. (0,-1)
      (0,-1) .. controls (-0.5,-0.8) and (-1,-0.4) .. (-1,0.3);
&
%Rüstungsklasse
\tikz [baseline = 0.6ex]
\draw [scale=0.7] (-1,1) -- (1,1) 
      (1,1) .. controls (1,-0.5) .. (0,-1)
      (0,-1) .. controls (-1,-0.5) .. (-1,1);
\\
Trefferpunkte & Rüstungsklasse \\
\end{tabular}

\end{minipage}
\begin{minipage}{.35\textwidth}

\vspace{.5cm}
\tikz\draw (0,0) -- (3,0);\linebreak
Dungeon Master

\tikz\draw (0,0) rectangle (4cm,4cm);\linebreak
Portrait oder Symbol des Charakters

\end{minipage}


\vspace{.5cm}

\begin{minipage}{.6\textwidth}

\tikz[baseline=0.6ex]\draw (0,0) rectangle (5ex,5ex); Stärke
 
\tikz[baseline=0.6ex]\draw (0,0) rectangle (5ex,5ex); Intelligenz

\tikz[baseline=0.6ex]\draw (0,0) rectangle (5ex,5ex); Weisheit

\tikz[baseline=0.6ex]\draw (0,0) rectangle (5ex,5ex);
Geschicklichkeit

\tikz[baseline=0.6ex]\draw (0,0) rectangle (5ex,5ex); Konstitution

\tikz[baseline=0.6ex]\draw (0,0) rectangle (5ex,5ex); Charisma 
 
\end{minipage}
\begin{minipage}{.4\textwidth}

\textbf{Rettungswürfe}

\begin{tabular}{cp{3.5cm}}
\tikz[baseline=0.6ex]\draw (0,0) circle (2.5ex); & Gift oder
Todesstrahlen \\ 

\tikz[baseline=0.6ex]\draw (0,0) circle (2.5ex); & Zauberstäbe \\

\tikz[baseline=0.6ex]\draw (0,0) circle (2.5ex); & Zu Stein erstarren
oder Lähmung \\

\tikz[baseline=0.6ex]\draw (0,0) circle (2.5ex); & Drachenodem \\

\tikz[baseline=0.6ex]\draw (0,0) circle (2.5ex); & Zaubersprüche,
-stecken oder -ruten \\

\end{tabular}

\end{minipage}

\textbf{Nahkampf}

\begin{tabular}{|l|c|c|c|c|c|c|c|c|c|c|c|c|c|}
\hline
RK & 9 & 8 & 7 & 6 & 5 & 4 & 3 & 2 & 1 & 0 & -1 & -2 \\
\hline
1W20 & & & & & & & & & & & & \\ 
\hline
\end{tabular}

\vspace{0.5cm}
\textbf{Fernkampf}

\begin{tabular}{|l|c|c|c|c|c|c|c|c|c|c|c|c|c|}
\hline
RK & 9 & 8 & 7 & 6 & 5 & 4 & 3 & 2 & 1 & 0 & -1 & -2 \\
\hline
1W20 & & & & & & & & & & & & \\ 
\hline
\end{tabular}

\textbf{Mitgeführte Ausrüstung}

\textbf{Allgemeine Notizen}

\begin{minipage}{0.5\textwidth}

\textbf{Geld und Schätze}

\end{minipage}
\begin{minipage}{0.5\textwidth}

\textbf{Erfahrung}

\vspace{3cm}
Gutschrift/Abzug: \hrulefill

\vspace{1cm}
Nächste Stufe wird erreicht bei: \hrulefill

\end{minipage}




\end{document}
